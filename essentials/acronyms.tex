% Set Acronym style
\newacronymstyle{single-table}
{%
    \GlsUseAcrEntryDispStyle{long-short}%
}
{%
    %\renewcommand*{\glossarysection}[2][]{}%
    \renewcommand{\glossentry}[2]{%
        \textbf{\glsentryshort{##1}} & \glsentrylong{##1} \newline \glsentrydesc{##1} \tabularnewline[3em]  % Spacing between entries
    }%
    \renewcommand{\glsgroupheading}[1]{}%
    \renewcommand{\glsgroupskip}{}%
    \renewenvironment{theglossary}%
    {\begin{longtable}{@{}p{0.15\linewidth}p{0.85\linewidth}@{}}}%
    {\end{longtable}\pagebreak}%
}
\setacronymstyle{single-table}

% Define Acronyms with description

\newacronym
[description={Univerzálny jazyk pre vizuálne modelovanie systémov}]
{UML}{UML}{Unified Modeling Language}

\newacronym
[description={Jazyk pre špecifikáciu obmedzení, dotazov a vyhľadávacích operácií v UML}]
{OCL}{OCL}{Object Constraint Language}

\newacronym
[description={Notácia pre modelovanie business procesov}]
{BPMN}{BPMN}{Business Process Model and Notation}

%Bioinformatika dvolezite akronymy pre nanopore sekvenovanie
\newacronym 
[description={Deoxyribonucleic Acid - DNA je molekula nesúca genetické informácie používané v rastlinách, zvieratách a väčšine organizmov. Je tvorená dvoma reťazcami nukleotidov, ktoré tvoria dvojitú špirálu.}]
{DNA}{DNA}{Deoxyribonucleic Acid}
\newacronym
[description={Ribonucleic Acid - RNA je molekula podobná DNA, ktorá hrá kľúčovú úlohu v procese prepisu a prekladu genetickej informácie. Na rozdiel od DNA je zvyčajne jednovláknová a obsahuje uracil namiesto thyminu.}]
{RNA}{RNA}{Ribonucleic Acid}
\newacronym
[description={Fast Quality - Formát súboru používaný na ukladanie sekvenčných dát spolu s kvalitatívnymi skóre pre každú bázu.}]
{FASTQ}{FASTQ}{Fast Quality}
\newacronym
[description={Signal-to-Noise Ratio - Poměr medzi úrovňou požadovaného signálu a úrovňou šumu, ktorý môže ovplyvniť kvalitu dát získaných z nanopórového sekvenovania.}]
{SNR}{SNR}{Signal-to-Noise Ratio}

