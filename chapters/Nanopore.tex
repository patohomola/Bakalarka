\typeout{TEST-START}
% Define a section
\section{Nanopore}
% Use an acronym

Nanopore je technológia používaná na sekvenovanie DNA a RNA molekúl. Táto metóda umožňuje čítanie genetického materiálu tým, že molekuly prechádzajú cez nanometrové póry, čo vedie k zmene elektrického prúdu, ktorý je následne analyzovaný na určenie sekvencie nukleotidov. 
Je to revolučná technológia v oblasti genomiky, ktorá umožňuje rýchle a presné sekvenovanie s minimálnymi nákladmi a zariadeniami prenosných veľkostí.
Aplikácie nanopórov zahŕňajú diagnostiku chorôb, výskum genetických porúch a monitorovanie environmentálnych vzoriek.

Nanopore sekvenovanie je obzvlášť užitočné pre jeho schopnosť čítať dlhé sekvencie DNA, čo zjednodušuje zostavovanie genómov a identifikáciu štruktúrnych variácií.
Funguje na princípe detekcie zmien v elektrickom prúde, keď molekula prechádza cez nanopór, čo umožňuje priame čítanie sekvencií bez potreby amplifikácie alebo značenia.
DNA alebo RNA molekuly sú vedené cez nanopór pomocou elektrického poľa, pričom každá báza spôsobuje charakteristickú zmenu prúdu, ktorá je zaznamenaná a analyzovaná softvérom na určenie sekvencie.

\subsection{Prekladanie DNa/Mra do signálu}

Prekladanie DNA alebo RNA sekvencií do elektrického signálu v nanopórovom sekvenovaní zahŕňa niekoľko kľúčových krokov:
\begin{itemize}
    \item \textbf{Príprava vzorky:} DNA alebo RNA molekuly sú pripravené na sekvenovanie, často zahŕňajúce fragmentáciu a pridanie adaptérnych sekvencií.
    \item \textbf{Vedenie cez nanopór:} Molekuly sú vedené cez nanopór pomocou elektrického poľa, ktoré spôsobuje ich pohyb.
    \item \textbf{Detekcia prúdu:} Keď molekula prechádza cez nanopór, každá báza spôsobuje charakteristickú zmenu v elektrickom prúde, ktorý je zaznamenaný ako časová séria dát.
    \item \textbf{Analýza signálu:} Zaznamenaný elektrický signál je analyzovaný pomocou algoritmov na identifikáciu sekvencie nukleotidov na základe zmien prúdu.
    \item \textbf{Preklad do sekvencie:} Softvér prekladá analyzovaný signál späť do sekvencie DNA alebo RNA, čo umožňuje výskumníkom získať genetické informácie z pôvodnej molekuly.
\end{itemize}

% vysvetli aky vystup dostaneme. A preco vela dna retazcov sa precita niekolkokrat.
Výstupom z procesu nanopórového sekvenovania je sekvencia nukleotidov (A, T, C, G pre DNA alebo A, U, C, G pre RNA) reprezentujúca genetickú informáciu pôvodnej molekuly. Tento výstup je často vo forme textového súboru (napríklad FASTQ formát), ktorý obsahuje sekvencie spolu s kvalitatívnymi skóre pre každú bázu. 
Vzhľadom na technické obmedzenia a variabilitu v procese sekvenovania je bežné, že jednotlivé DNA alebo RNA retiazce sú prečítané niekoľkokrát (tzv. "coverage" alebo "depth of coverage"). Toto opakované čítanie zvyšuje spoľahlivosť a presnosť výslednej sekvencie, pretože umožňuje korekciu chýb a identifikáciu variácií v genetickom materiáli.
No napriek tomu, že sa retiazce čítajú viackrát, stále môže dôjsť k nepresnostiam v sekvencii kvôli šumu v signáli.

% vysvetli ako dna cita signal. za rovnako precitane dna retazce mozu mat rozdielny signal.

% Ako mozu pri citani dna vzniknut chyby.

% Dna sa neskopiruje dokonale.

% dna sa cita rozne rychlo.
Retazce DNA alebo RNA môžu byť čítané rôznou rýchlosťou počas nanopórového sekvenovania, čo môže ovplyvniť kvalitu a presnosť zaznamenaného signálu. Rýchlosť prechodu molekuly cez nanopór je ovplyvnená viacerými faktormi, vrátane veľkosti molekuly, jej konformácie, interakcií s nanopórom a podmienok prostredia (napríklad teplota a iónová sila).

%Dna je velmi male a preto dostaneme signal pre kazdu katicu v dna molekule.
Keďže DNA a RNA molekuly sú veľmi malé, signál zaznamenaný počas nanopórového sekvenovania odráža interakcie jednotlivých nukleotidov (báz) s nanopórom. Každá báza má jedinečný tvar a elektrické vlastnosti, ktoré ovplyvňujú spôsob, akým mení elektrický prúd pri prechode cez nanopór. Preto je možné získať signál pre každú jednotlivú bázu v molekule, čo umožňuje presné čítanie sekvencie.