% vipis vsetky vyhody ma ukladania dat do DNA
\section{Motivácia}
V dnešnej digitálnej dobe sa množstvo generovaných dát exponenciálne zvyšuje,
čo vytvára potrebu efektívnych a trvalo udržateľných metód ich ukladania. 
Tradičné úložné médiá, ako sú pevné disky, SSD disky a magnetické pásky,
majú obmedzenú životnosť a kapacitu, čo vedie k častým migráciám dát
a zvýšeným nákladom na údržbu. Naopak, DNA ako médium pre ukladanie dát ponúka niekoľko významných výhod:
\begin{itemize} 
    \item \textbf{Vysoká hustota ukladania:} DNA má neuveriteľne vysokú hustotu ukladania dát, kde jeden gram DNA môže teoreticky uložiť až 215 petabajtov (215 miliónov gigabajtov) dát. To znamená, že obrovské množstvo informácií môže byť uložené v mimoriadne malom objeme.
    \item \textbf{Dlhodobá stabilita:} DNA je chemicky stabilná molekula, ktorá môže prežiť tisíce rokov za správnych podmienok. Na rozdiel od tradičných médií, ktoré sa môžu degradovať počas niekoľkých dekád, DNA môže uchovávať dáta po veľmi dlhú dobu bez straty integrity.
    \item \textbf{Energetická efektívnosť:} Ukladanie dát do DNA nevyžaduje neustálu energiu na udržiavanie dát, na rozdiel od elektronických úložísk, ktoré potrebujú napájanie na zachovanie informácií. To vedie k výrazným úsporám energie a znižuje ekologickú stopu dátových centier.
    \item \textbf{Ľahká replikácia a prenosnosť:} DNA môže byť ľahko kopírovaná pomocou biologických procesov, čo umožňuje jednoduché zálohovanie a prenos dát medzi rôznymi miestami bez potreby špecializovaného hardvéru.
    \item \textbf{Odolnosť voči technologickému zastaraniu:} Na rozdiel od tradičných úložných médií, ktoré môžu rýchlo zastarať v dôsledku technologického pokroku, DNA ako médium zostáva nezmenené a čitateľné pomocou základných biologických techník.
\end{itemize}


Napriek týmto výhodám, proces čítania DNA pomocou nanopórov prináša špecifické výzvy.
Pri sekvenovaní DNA cez nanopór sa DNA molekula prechádza cez nanopór po jednotlivých nukleotidoch,
pričom elektrický signál identifikuje bázy. Tento proces je však náchylný na chyby, 
keďže DNA sa môže pohybovať rýchlo alebo nepravidelne, čo vedie k nesprávnemu
určeniu sekvencie nukleotidov. Tieto chyby v čítaní majú priamy vplyv na integritu
uložených dát a vyžadujú sofistikované metódy kódovania a korekcie chýb.
Avšak s pokračujúcim vývojom technológií nanopórového sekvenovania sa
presnosť postupne zvyšuje, čo robí z DNA perspektívnu alternatívu k tradičným metódam ukladania dát.

% vysvetli ze mojim cielom je pomoct pri zefektivneni procesu ukladania dat do dna a to tak ze vytvorim dna kanal s lepsimi vlastnostami pre citanie a zapis dat do dna.
Cieľom tejto práce je prispiet k zefektívneniu procesu ukladania 
dát do DNA vytvorením DNA kanála s lepšími vlastnosťami pre čítanie a zápis dát.
Optimalizáciou týchto procesov môžeme zvýšiť rýchlosť a presnosť ukladania a načítavania informácií, 
čo je kľúčové pre praktické využitie DNA ako úložného média v budúcnosti.

% vmozes my vysvetlit co je to dna kanal?
DNA kanál predstavuje špecifický spôsob, akým sú dáta kódované a ukladané do DNA molekúl.
Tento kanál zahŕňa výber vhodných sekvencií nukleotidov, ktoré minimalizujú chyby pri čítaní 
a zápise dát,
ako aj optimalizáciu procesov syntézy a sekvenovania DNA. 

