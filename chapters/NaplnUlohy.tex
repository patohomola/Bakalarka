\typeout{TEST-START}
% Define a section
\subsection{Uloha}
% Use an acronym

% vysvetli mundane ze cielom je simulovat ake chybi vznikaju pri citani dna pomocou nanopore sekvencovania.
Cieľom tejto úlohy je simulovať chyby, ktoré vznikajú pri čítaní DNA pomocou nanopórového sekvenovania. Nanopórové sekvenovanie je inovatívna technológia, ktorá umožňuje rýchle a efektívne čítanie genetického materiálu. 
%Priklad dostanes  retazec realneho dna retazca a ja vytvorim generator ktory nahodne pridava chyby do dna retazca na zaklade pravdepodobnosti chyby.
Úlohou je vytvoriť generátor, ktorý na základe pravdepodobnosti chyby náhodne pridáva chyby do reálneho DNA reťazca. Tento generátor by mal byť schopný simulovať rôzne typy chýb, ako sú záměny báz, vloženia alebo vynechania báz, ktoré sa môžu vyskytnúť počas procesu sekvenovania. 

%sluzit to bude na testovanie a vyhodnocovanie algoritmov pre spracovanie sekvenačných dát, ktoré musia byť robustné voči týmto chybám.
Táto simulácia bude slúžiť na testovanie a vyhodnocovanie algoritmov pre spracovanie sekvenačných dát, ktoré musia byť robustné voči týmto chybám. Cieľom je zlepšiť presnosť a spoľahlivosť analýzy genetických informácií získaných pomocou nanopórového sekvenovania.
