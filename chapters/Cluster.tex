\section{Cluster clover}

Klastrovací algoritmus je metóda, ktorá rozdeľuje množinu dát do skupín (klastrov) tak, aby dáta v rámci jednej skupiny boli si navzájom podobné a dáta z rôznych skupín boli odlišné \cite{Qu2022Clover}. Tieto algoritmy sa často používajú v bioinformatike na analýzu sekvencií DNA alebo iných biologických dát.

Clover algoritmus je efektívny klastrovač navrhnutý špeciálne pre DNA sekvencie v oblasti DNA-based data storage. Využíva stromovú štruktúru na rýchle a presné zoskupovanie sekvencií podľa ich podobnosti, čím zvyšuje efektivitu a škálovateľnosť procesu \cite{Qu2022Clover}. Clover umožňuje efektívne spracovanie veľkých množín sekvencií a je vhodný pre moderné aplikácie v oblasti ukladania dát do DNA.
