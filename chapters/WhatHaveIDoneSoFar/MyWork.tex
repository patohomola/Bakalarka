
Najskôr som stiahol potrebný dataset a pripravil prostredie Dorado, ktoré je určené na spracovanie sekvenačných dát.





Následne som spustil basecaller Dorado, ktorým som preložil súbory vo formáte pod5 do formátu BAM, čo umožňuje ďalšie spracovanie dát.


Potom som použil algoritmus Clover, ktorý zoskupil sekvencie do jednotlivých klastrov na základe ich podobnosti.


Zo všetkých vytvorených klastrov som vybral tie, ktoré boli dostatočne veľké na ďalšiu analýzu.


Na každý vybraný cluster som aplikoval nástroj Medaka, aby som vytvoril konsenzuálnu sekvenciu pre daný cluster.


Pre každú katicu (k-mer) z konsenzuálnej sekvencie som vybral reprezentatívnu katicu z alignovaných sekvencií v danom clustri.


Nakoniec som všetky získané dáta uložil pre ďalšie spracovanie alebo analýzu.
